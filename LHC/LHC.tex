\chapter{Le Grand collisionneur de hadrons (LHC)}
\renewcommand\chapterillustration{LHC/lhc}
\ThisULCornerWallPaper{1}{\chapterillustration}
\minitoc
\lettrine[lines=4, slope=-0.5em]{C}{e} chapitre décrit le complexe des accélérateurs du CERN\footnote{Organisation Européenne pour le Recherche Nucléaire (Laboratoire européen de physique des particules)} qui permet d'accélérer les particules, afin d'avoir un faisceau de particules avec une énergie suffisante pour être injecté dans le Grand Collisionneur de Hadrons (LHC\footnote{Large Hadron Collider}) et atteindre une énergie finale de $7$ TeV. Cette description bien que succincte est nécessaire car de nombreux résultats obtenus durant cette thèse ont nécessité l'utilisation d'accélérateurs de ce complexe. Une description du LHC est également donnée, car ses performances présentes et futures déterminent les choix technologiques des détecteurs utilisant son faisceau.

\section{Le complexe d'accélérateurs du CERN}

Le complexe d'accélération (\ref{complexe}) du CERN est une série de machines qui délivrent des faisceaux de particules d'énergies de plus en plus élevées. Chaque machine accélère les faisceaux et les injecte dans la machine suivante. Le dernier accélérateur du complexe est le LHC.

Le programme de ce dernier est surtout basé sur des collisions protons-protons. Cependant chaque année, environ un mois est consacré aux collisions d'ions lourds (plomb-plomb) ou (proton-plomb) afin d'étudier notamment le plasma de quarks et gluons, l'une des phases de l'Univers peu après le Big Bang. 

Dans le cas de ces collisions, la chaine d'accélération est constituée du Linear Accelerator 3 (LINAC 3), du Low Energy Ion Ring (LEIR) utilisé pour le stockage des ions et leur refroidissement. La chaine d'accélération est ensuite identique à celle pour les collisions proton-proton.

\begin{minipagewithmarginpars}[h]{\textwidth}
  	\centering
	\includegraphics[scale=0.40]{LHC/complexe.png}
  	\captionof{figure}{Schéma du complexe d'accélération du CERN. La chaine d'injection du LHC est constituée du Linac 2, du Booster, du PS et du SPS}
  	\label{complexe}
  	\par 	
\marginpar
{
	\includegraphics[width=\marginparwidth]{LHC/Bouteille.jpg}
	\label{bouteille}
    	\captionof{figure}{Source des protons du LHC.}
}	
\end{minipagewithmarginpars}

\marginpar
{
	\includegraphics[width=\marginparwidth]{LHC/linac2.jpg}
    \captionof{figure}{Photo du LINAC 2.}
    	\label{linac2}
}

\marginpar
{
	
	\includegraphics[width=\marginparwidth]{LHC/booster.jpg}
    \captionof{figure}{Photo du Booster du Synchrotron à protons.}
    	\label{booster}
}

\marginpar
{
	
	\includegraphics[width=\marginparwidth]{LHC/ps.jpg}
    \captionof{figure}{Photo du PS.}
    	\label{ps}
}

\marginpar
{
	
	\includegraphics[width=\marginparwidth]{LHC/sps.jpg}
    \captionof{figure}{Photo du SPS.}
    	\label{sps}
}
Pour les collisions proton-proton, la source de protons est une bouteille de dihydrogène gazeux (fig. \ref{bouteille}). Les atomes d’hydrogène sont injectés dans le Duoplasmatron Proton IOn Source où il sont chauffés et sont soumis à un champ électrique, qui arrache leurs électrons et les ionise en $H^{+}$ (proton). Les protons extraits sont ensuite envoyés dans l'accélérateur linaire (LINAC 2 (fig. \ref{linac2})) où ils atteignent l'énergie de 50 MeV et sont $5\%$ plus massifs, leur vitesse est alors d'environ $0.3$c où c est la vitesse de la lumière dans le vide (c=$299 792 458$ m/s). Ils passent ensuite dans les 4 anneaux de 157m de circonférence du Booster du Synchrotron à protons (BOOSTER (fig. \ref{booster})) qui les amènent à une énergie de 1.4 GeV avant de les injecter dans l'accélérateur suivant, le Synchrotron à protons (PS (fig. \ref{ps})). Cet accélérateur circulaire de 628 mètres de circonférence, permet aux faisceaux d'atteindre une énergie de 25 GeV, leur vitesse est alors de 0.87c. Il sert aussi à préparer le faisceau en le découpant en série de paquets (bunchs) de particules nécessaires au LHC. Ces bunchs sont ensuite envoyés dans le supersynchrotron à protons (SPS (fig. \ref{sps})) d'une circonférence de 7 km, où l'énergie du faisceau atteint 450 GeV soit une vitesse de 0.99c. Les paquets sont regroupés pour former des trains de paquets avant d'être enfin envoyés dans le Grand Collisionneur de Protons (LHC). L'injection et le guidage de faisceaux d'une telle énergie par des aimants supraconducteurs rapides est une tâche délicate et pourrait détériorer l'accélérateur. Un faisceau de test de faible intensité "pilot beam" est donc injecté afin de mesurer et vérifier les paramètres. Le faisceau de haute énergie est ensuite séparé en deux et injecté dans deux conduits différents, l'un circulant dans un sens et l'autre dans le sens contraire. Ces faisceaux sont ensuite accélérés jusqu'à une énergie de 7 TeV et ne se croisent qu'aux points d'intéractions. À ce stade, les protons on une vitesse de 0.999999991c, soit 299 792 455,3 m/s. Il ne se déplace que 2.7m/s moins vite que la lumière. Afin d'accélérer les protons à une vitesse si proche de la lumière, d'importantes contraintes de pressions et de température sont nécessaires. Un vide poussé est nécessaire à l'intérieur du LHC afin de minimiser les interactions, la pression interne est de l'ordre de $10^{-13}$atm soit 10fois moins que la pression à la surface de la Lune. La présence d'aimants supraconducteurs nécessite un système de distribution cryogénique, qui assure la circulation d'hélium super-fluide est maintient le LHC à une température de $-271,3$\degre C ($1.9$K). Le LHC est donc plus froid que l'Univers (Le fond diffus cosmologique à en effet été évalué par le satellite Plank (cf.fig\ref{Plank}) (2009-2013) à $2,725$K.

\section{Le Large Hadron Collider}
Le LHC est le dernier accélérateur circulaire du complexe d'accélération. Il utilise le tunnel de 27 km de circonférence situé à une centaine de mètres sous terre. Il fût construit pour acceuillir le Grand collisionneur électron-positron (LEP\footnote{Large Electron Positron collider.}), qui fût en service de 1989 à 2000. Le LHC à été mis en service en 2008 et a été construit afin de produire de l'ordre de 600 millions de collisions proton-proton par seconde à une énergie au centre de masse de $\sqrt{s}=14$ TeV. Il est actuellement l'accélérateur de proton-proton le plus puissant du monde, et a permis de mettre en évidence l'existence du boson de Higgs, dernière pièce manquante du Modèle Standard.

Le LHC est un collissioneur de particules non fondamentales (hadrons) à l'inverse de son prédécesseur, le LEP qui utilisé des électrons et des positrons. Lors d'une collission entre hadrons, ceux sont ses constituant élementaires, les quarks et les gluons qui collissionnent entre eux. Ceux-ci possèdent seulement une portion de l'énergie du hadrons qui les contient. L'énergie du centre de masse de cette collision n'est donc pas connue avec précision. Le LHC est donc une machine de découverte de particules plutôt qu'une machine de mesures de précisions comme l'était le LEP, car il permet d'acceder à un large spectre en énergie. Généralement, les mesures de précision sont effectué grâce à des collisionneur utilisant des particules élémentaires ($e^{-}$,$e^{+}$); ils sont dans ce cas souvent linéaire afin d'éviter la perte d'énergie par rayonnement synchroton.

La figure \ref{lhcschema} est une vue schématique du LHC. En vérité le LHC n'est pas parfaitement circulaire, mais est composé de $8$ octants composés d'une section droite de longueur $~5$ km et d'un secteur courbe d'une longueur de $~3$ km (fig. \ref{octants}). Afin de courber les faisceaux 1232 aimants dipolaires de 15 mètres de long sont utilisés et 392 aimants quadripolaires de 5 à 7 mètres servent à concentrer les faisceaux. Les sections droites sont utilisées afin de faire collisionner les deux faisceaux de protons venant en sens inverse l'un de l'autre. Il existe $8$ points potentiels d'interactions (P), mais seulement $4$ sont le siège de collisions et possèdent des détecteurs qui analysent les données issues de ces collisions : le point P1 pour ATLAS\footnote{A Toroidal LHC ApparatuS, détecteur généraliste.}, le point P2 pour ALICE\footnote{A Large Ion Collider Experiment, dédié à l'étude du plasma de quarks et gluons.}, le point P5 pour CMS\footnote{Compact Muon Solenoid, détecteur généraliste.} et le point P8 pour LHCb\footnote{Large Hadron Collider beauty experiment, dédié au quark b.}.

\marginpar
{
	\includegraphics[width=\marginparwidth]{LHC/atlas.png}
    \captionof{figure}{ATLAS.}
    	\label{atlas}
}
\marginpar
{
	\includegraphics[width=\marginparwidth]{LHC/alice.png}
    \captionof{figure}{ALICE.}
    	\label{alice}
}
\marginpar
{
	\includegraphics[width=\marginparwidth]{LHC/cms.png}
    \captionof{figure}{CMS.}
    	\label{cms}
}
\marginpar
{
	
	\includegraphics[width=\marginparwidth]{LHC/lhcb.png}
    \captionof{figure}{LHCb.}
    	\label{lhcb}
}

\begin{minipagewithmarginpars}[h]{\textwidth}
  	\centering
	\includegraphics[scale=0.7]{LHC/CERNMap.jpg}
  	\captionof{figure}{Vue schématique du LHC.}
  	\label{lhcschema}	
\end{minipagewithmarginpars}

\begin{minipagewithmarginpars}[h]{0.95\textwidth}
  	\centering
	\includegraphics[width=0.55\textwidth]{LHC/lhc-schematic.jpg}
  	\captionof{figure}{Vue schématique des octants du LHC ainsi que des positions des principaux détecteurs le long du LHC. Les faisceaux (en bleu et rouge) circulent en sens inverse l'un de l'autre.}
  	\label{octants}	
\end{minipagewithmarginpars}
La réutilisation du tunnel du LEP et l'énergie des faisceau (7 TeV) à obligé l'utilisation d'aimant supraconducteur produsant des champs magnétique de 8.4T. De plus l'utilisation de deux faisceaux de protons voyant en sens inverse et l'espace limité dans le tunnel à amené à la création d'un nouveau type d'aimants où les deux tubes contenant les deux faisceaux sont insérés dans un même cryostat (cf.fig\ref{dipole}).

\begin{minipagewithmarginpars}[h]{0.95\textwidth}
\centering
\includegraphics[width=0.95\textwidth]{LHC/dipole.jpg}
\captionof{figure}{Vue en transversale d'un dipole du LHC.}
\label{dipole}	
\end{minipagewithmarginpars}

\section{Luminosité des faisceaux}
Le nombre d'événement correspondant à un processus donnée que peut produire le LHC peut être exprimé par :
\begin{equation}
N=\int \mathcal{L}(t)\sigma \mathrm dt
\end{equation}
où $\sigma$ est la section efficace du processus considéré ( la probabilité q'un événement produise ce processus ), $t$ la durée de prise de donnée et $\mathcal{L}$ la luminosité instantanée délivré par la machine.

La luminosité est déterminée par les paramètres des faisceaux de protons :
\begin{equation}
\mathcal{L}=\frac{f_{rev}n_{p}N_{}}{4\pi \sigma_{x} \sigma_{y}} F(\phi)
\end{equation}
avec $N_{f}$ le nombre de protons par paquet, $n_{p}$ le nombre de paquets, $f_{rev}$ la fréquence de rotation d'un paquet, $\sigma_{x}$,($\sigma_{y}$) la moyenne quadratique horizontale (verticale) transverse de la taille du faisceau au point d'intéraction et $F(\phi)$ est le facteur de réduction géométrique définit en fonction de l'angle $\phi$ de Piwinski.
\begin{equation}
F(\phi)=\frac{1}{\sqrt{1+\phi^{2}}}
\end{equation}
En considérant des faisceau se croisant dans le plan verticale (cf.fig\ref{collision}), l'angle de Piwinski peut s'écrire :
\begin{equation}
\phi=\frac{\theta_{c}\sigma_{z}}{2\sigma_{y}}
\end{equation}
avec $\theta_{c}$ l'angle de croisement des faisceaux, $\sigma_{z}$ la moyenne quadratique de la longueur d'un paquet.


\begin{minipagewithmarginpars}[h]{0.95\textwidth}
\centering
\includegraphics[width=0.5\textwidth]{LHC/collision.png}
\captionof{figure}{Schéma de colisions de paquets dans le plan transversale.}
\label{collision}	
\end{minipagewithmarginpars}

Il est possible d'exprimer $\sigma_{x}$ et $\sigma_{y}$ en fonction de l'émittance $\epsilon$ et de la fonction $\beta(s)$. L'émittance est l'espace dans l'espace des phase qui contientun certain pourcentage des particules du faisceau ( 95\% pour les collisionneur hadroniques). Elle est supposée constante durant toute la durée du temps de mesure \footnote{d'après le théorème de Liouville} :
\begin{equation}
\sigma_{x}=\sqrt{\epsilon\beta_{x}}
\end{equation}
avec s la postion le long de la trajectoire nominale du faisceau.

La valeur de la fonction $\beta$ au point d'interaction est noté $\beta^{*}$.En considérant de plus que $\sigma_{x}=\sigma_{x}=\sigma$ on peut réécrire $\mathcal{L}$ comme :
\begin{equation}
\mathcal{L}=\frac{f_{rev}\gamma\beta n_{p}N_{p}}{4\epsilon_{n}\beta^{*}\pi} F(\phi)
\end{equation}
où $\epsilon_{n}=\epsilon\gamma\beta$ est l'émittance normalisé, $\gamma$ le facteur de Lorentz $\beta=v/c$.

D'après cette formule on peut remarqué qu'il peut être interéssant de réduire au maximun $\epsilon_{n}$ c'est à dire d'avoir des paquet dont les particules ont la même quantité de mouvement et son proche l'une de l'autre. Il est aussi possible de réduire le facteur de réduction géométrique en utilisant une configuration de faisceau dite en "crabe" (cf.fig\ref{crabe})

\begin{minipagewithmarginpars}[h]{0.95\textwidth}
\centering
\includegraphics[width=0.5\textwidth]{LHC/crab.png}
\captionof{figure}{Schéma de colisions "en crab" de paquets dans le plan transversale.}
\label{crabe}	
\end{minipagewithmarginpars}

La luminosité du faisceau ne reste pas constante tout le long d'un cycle de prise de donnée. La cause principale sont les collisions dans les détecteurs.Le temps de décroissance caractéristique de l'intensité du faisceau par les collisions est :
\begin{equation}
\tau{collisions}=\frac{N_{0}}{L_{0}\sigma_{tot}n}
\end{equation}
avec $N_{0}$ le nombre de protons à l'injection, $L_{0}$ la luminosité initiale,$\sigma_{tot}$ la section efficace totale et $n$ le nombre de point d'interactions des faisceaux.

Les deux autres causes principales de la réduction de la luminosité des faisceaux sont d'une part la perte par interaction entres les particules du faisceaux et le gaz piégé dans les tubes (avec un temps caractéristique $\tau_{gaz}$) ainsi que par diverses perturbations et imperfections ( champ magnétique par exemple ) qui peuvent dévier les particules de la trajectoire nominale du faisceau ($\tau_{imper}$) . 
La décroissance de la luminosité instantanée peut être donné par la formule:
\begin{equation}
\frac{\mathrm \mathcal{L}}{\mathrm t} = L_{0} e^{-\frac{t}{\tau}}
\end{equation}
avec
\begin{equation}
\frac{1}{\tau} = \frac{1}{\tau_{collisions}}+\frac{1}{\tau_{gaz}}+\frac{1}{\tau_{imper}}
\end{equation}

Lorsque l'intensité du faisceau devient trop faible pour une prise de données efficace, les faisceaux sont déviés de leurs trajectoires circulaires et envoyé dans les "dump blocks" (cf.fig\ref{dump})
\marginpar
{
	\includegraphics[width=\marginparwidth]{LHC/dump.png}
    \captionof{figure}{Schéma d'un "dump block".}
    	\label{dump}
}

La luminosité intégré pour les collisions protons-protons délivré par le LHC en 2016 ainsi que celle enregistré par  le détecteur CMS est donné par le graphique suivant :
\begin{minipagewithmarginpars}[h]{0.95\textwidth}
\centering
\includegraphics[width=0.7\textwidth]{LHC/luminosity.png}
    \captionof{figure}{Luminosité intégrée en fonction du jour de l'année 2016 délivrée (bleu) et enregistrée par CMS (orange) pendant les faisceaux stables et pour les collisions pp à 13 TeV d'énergie dans le centre de masse en 2016.}
\end{minipagewithmarginpars}
