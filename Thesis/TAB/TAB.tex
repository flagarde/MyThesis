\chapter{Table récapitulative des tests en faisceaux}
\label{tableau}
\renewcommand\chapterillustration{TAB/TAB}
\ThisULCornerWallPaper{1}{\chapterillustration}
\vspace*{-0.7cm}
\lettrine[lines=2, slope=-0.5em]{C}{ette} table répertorie les différent test en faisceaux effectués et présente également les électroniques, chambres et faisceaux utilisés ainsi que les informations utiles pour chacun de ces tests.

\begin{table}
	\vspace*{-18.5cm}
	\centering
\begin{tabular}{|O|O|O|O|N}
	\hline 
	\rowcolor{gray!50}Test& DESY & PS& SPS \\ 
	\hline 
	\thead{Section de la thèse}& \ref{DESYY} (p.\pageref{DESYY}) & \ref{PSS} (p.\pageref{PSS}) & \ref{SPSS} (p.\pageref{SPSS})\\ 
	\hline 
	Date & \thead{octobre 2014}& juin 2014 & juin 2015   \\ 
	\hline 
	Lieu & \thead{DESY \\Allemagne} & \thead{PS \\T9 CERN} & \thead{PS \\H2 CERN}  \\ 
	\hline 
	\thead{Type de faisceaux}& $e^{-}$ (continu)& $\mu$ (\textit{spill}) & $\mu$ (\textit{spill}) \\ 
	\hline 
	\thead{Intensité  de faisceaux}& \SI{9}{\kilo\hertz\per\square\centi\meter}& \SI{10}{\kilo\hertz\per\square\centi\meter} & $\sim$\SI{8}{\kilo\hertz\per\square\centi\meter}  \\ 
	\hline 
	\thead{Type de détecteur}& \thead{\textit{float glass}\\Basse résistivité}& \thead{\textit{float glass}\\Basse résistivité} &\thead{\textit{float glass}\\Basse résistivité}    \\ 
	\hline 
	\thead{Taille des détecteurs}& \num{32}$\times$\num{30}\si{\square\centi\meter}& \num{32}$\times$\num{30}\si{\square\centi\meter} & \num{32}$\times$\num{30}\si{\square\centi\meter}   \\ 
	\hline 
	\thead{Nombre de détecteurs}&1/4 & 5/3 & 4/1   \\ 
	\hline 
	\thead{Épaisseur des électrodes}&\SI{1}{\milli\meter} & \SI{1}{\milli\meter} &  \SI{1}{\milli\meter}\\ 
	\hline 
	\thead{Épaisseur du \textit{gap} }&\SI{1.2}{\milli\meter} & \SI{1.2}{\milli\meter}&\SI{1.2}{\milli\meter} \\ 
	\hline 
	Gaz &ILC & ILC & ILC   \\ 
	\hline 
	\thead{Type de chambre}& \textit{single gap} & \thead{7 \textit{single gap} \\ 1 \textit{double gaps}} & \textit{single gap}  \\ 
	\hline 
	\thead{Type d'électronique}& HARDROC & HARDROC & HARDROC   \\ 
	\hline 
	\thead{Type de cellules \\de lecture}& \textit{Pad} & \thead{7\textit{Pad}\\1 \textit{Strip} (cf.Fig~\ref{DoubleGap})} & \textit{Pad}   \\ 
	\hline
	\thead{Tension de fonctionnement}& \SI{7.2}{\kilo\volt} & \SI{7.2}{\kilo\volt} & \SI{6.9}{\kilo\volt}  \\ 
	\hline
	\thead{Seuil de fonctionnement}& \SI{50}{\femto\coulomb} & \SI{0.13}{\pico\coulomb} & \SI{0.13}{\pico\coulomb}  \\ 
	\hline
\end{tabular} 
\caption{Table répertoriant les différents tests en faisceaux.}
\end{table}

\captionsetup{list=no}
\begin{table}
	\centering
\begin{tabular}{|O|O|O|O|N}
	\hline 
	\rowcolor{gray!50}Test& \thead{GIF++\\août 2015}& Vieillissement & \thead{GIF++\\mai-juin 2016}\\ 
	\hline 
	\thead{Section de la thèse}& \ref{GIFFF} (p.\pageref{GIFFF}) & \ref{VIEE} (p.\pageref{VIEE}) & \ref{GIFFF2} (p.\pageref{GIFFF2}) \\ 
	\hline 
	Date & août 2015 & août--octobre 2015 & mai--juin 2016  \\ 
	\hline 
	Lieu & \thead{GIF++ \\H4 CERN} & \thead{GIF++ \\H4 CERN} & \thead{GIF++ \\H4 CERN}  \\ 
	\hline 
	\thead{Type de faisceaux}& $\mu$ (\textit{spill})& --- & $\mu$ (\textit{spill})  \\ 
	\hline 
	\thead{Intensité  de faisceaux}& $\sim$\num{1000}/spill& --- & $\sim$\num{1000}/spill   \\ 
	\hline 
	\thead{Type de détecteur}& \thead{\textit{float glass}\\Basse résistivité}& \thead{\textit{float glass}\\Basse résistivité} &\thead{\textit{float glass}\\Basse résistivité}   \\ 
	\hline 
	\thead{Taille des détecteurs}& \num{32}$\times$\num{30}\si{\square\centi\meter}& \num{32}$\times$\num{30}\si{\square\centi\meter} & \num{32}$\times$\num{30}\si{\square\centi\meter}  \\ 
	\hline 
	\thead{Nombre de détecteurs}&3/4 & 3(1)/4 & 1/4  \\ 
	\hline 
	\thead{Épaisseur des électrodes}&\SI{1}{\milli\meter} & \SI{1}{\milli\meter} & \SI{1}{\milli\meter} \\ 
	\hline 
	\thead{Épaisseur du \textit{gap} }&\SI{1.2}{\milli\meter} & \SI{1.2}{\milli\meter}& \SI{1.2}{\milli\meter} \\ 
	\hline 
	Gaz &CMS & CMS & CMS   \\ 
	\hline 
	\thead{Type de chambre}& \textit{single gap} & \textit{single gap} & \textit{single gap}  \\ 
	\hline 
	\thead{Type d'électronique}& HARDROC & HARDROC & HARDROC \\ 
	\hline 
	\thead{Type de cellules \\de lecture}& \textit{Pad} & \textit{Pad} & \textit{Pad}   \\ 
	\hline
	\thead{Tension de fonctionnement}& \SI{7.0}{\kilo\volt} & variable & \SI{6.9}{\kilo\volt}  \\ 
	\hline
	\thead{Seuil de fonctionnement}& \SI{0.13}{\pico\coulomb} & \SI{0.13}{\pico\coulomb} & \SI{0.13}{\pico\coulomb}   \\ 
	\hline
\end{tabular} 
\addtocounter{table}{-1}
\renewcommand{\thetable}{A.\arabic{table} (suite)}
\caption{Table répertoriant les différents tests en faisceaux.}
\end{table}


\begin{table}
	\centering
\begin{tabular}{|O|O|O|N}
	\hline 
	\rowcolor{gray!50}Test&banc de tests& \thead{SPS \\mai--juin 2016}\\ 
	\hline 
	\thead{Section de la thèse}& \ref{BANCC} (p.\pageref{BANCC})&  \ref{SPSS2} (p.\pageref{SPSS2})\\ 
	\hline 
	Date & mars 2016& mai--juin 2016\\ 
	\hline 
	Lieu & \thead{Lyon} & \thead{SPS \\H2 CERN}  \\ 
	\hline 
	\thead{Type de faisceaux}& Cosmique& $\mu$ (\textit{spill})  \\ 
	\hline 
	\thead{Intensité  de faisceaux}&$\sim$\SI{100}{\per\square\meter\per\second\per\steradian}&  $\sim$\num{1000}/spill\\ 
	\hline 
	\thead{Type de détecteur}&\thead{\textit{float glass}\\Collage/Fixation mécanique}& \thead{\textit{float glass}\\ basse résistivité} \\ 
	\hline 
	\thead{Taille des détecteurs}& RE1/1& RE1/1\\ 
	\hline 
	\thead{Nombre de détecteurs}&2& 2 \\ 
	\hline 
	\thead{Épaisseur des électrodes}&\SI{1}{\milli\meter} & \SI{1}{\milli\meter} \\ 
	\hline 
	\thead{Épaisseur du \textit{gap} }&\SI{1.2}{\milli\meter} & \SI{1.2}{\milli\meter} \\ 
	\hline 
	Gaz &ILC& ILC\\ 
	\hline 
	\thead{Type de chambre}& \textit{double gaps} & \textit{double gaps}  \\ 
	\hline 
	\thead{Type d'électronique}& HARDROC & FEB \\ 
	\hline 
	\thead{Type de cellules \\de lecture}& \textit{strip} & \textit{strip}  \\ 
	\hline
	\thead{Tension de fonctionnement}& --- & \SI{7.0}{\kilo\volt}  \\ 
	\hline
	\thead{Seuil de fonctionnement}& \SI{0.13}{\pico\coulomb} & $\sim$\SI{300}{\milli\volt}   \\ 
	\hline
\end{tabular} 
\addtocounter{table}{-1}
\renewcommand{\thetable}{A.\arabic{table} (suite)}
\caption{Table répertoriant les différents tests en faisceaux.}
\end{table}

\begin{table}
	\centering
\begin{tabular}{|O|O|O|O|O|O|O|O|N}
	\hline 
	\rowcolor{gray!50}Test&\thead{GIF++\\août 2016} &\thead{GIF++ \\octobre 2016}\\ 
	\hline 
	\thead{Section de la thèse}& \ref{GIFFF3} (p.\pageref{GIFFF3})& \ref{GIFFF4} (p.\pageref{GIFFF4})   \\ 
	\hline 
	Date & août 2016& octobre 2016   \\ 
	\hline 
	Lieu & \thead{GIF++ \\H4 CERN} & \thead{GIF++ \\H4 CERN}   \\ 
	\hline 
	\thead{Type de faisceaux}& $\mu$ (\textit{spill})& $\mu$ (\textit{spill})  \\ 
	\hline 
	\thead{Intensité  de faisceaux}&$\sim$\num{1000}/spill&  $\sim$\num{1000}/spill  \\ 
	\hline 
	\thead{Type de détecteur}&\thead{Basse résistivité\\Fixation mécanique}& \thead{Basse résistivité\\Fixation mécanique} \\ 
	\hline 
	\thead{Taille des détecteurs}& RE1/1& RE1/1\\ 
	\hline 
	\thead{Nombre de détecteurs}&1& 1  \\ 
	\hline 
	\thead{Épaisseur des électrodes}&\SI{1}{\milli\meter} & \SI{1}{\milli\meter}\\ 
	\hline 
	\thead{Épaisseur du \textit{gap} }&\SI{1.2}{\milli\meter} & \SI{1.2}{\milli\meter} \\ 
	\hline 
	Gaz &CMS& CMS \\ 
	\hline 
	\thead{Type de chambre}& \textit{double gaps} & \textit{double gaps}   \\ 
	\hline 
	\thead{Type d'électronique}& FEB & FEB   \\ 
	\hline 
	\thead{Type de cellules \\de lecture}& \textit{strip} & \textit{strip}  \\ 
	\hline
	\thead{Tension de fonctionnement}& \SI{7.0}{\kilo\volt} & \SI{7.0}{\kilo\volt}   \\ 
	\hline
	\thead{Seuil de fonctionnement}&  $\sim$\SI{300}{\milli\volt}& $\sim$\SI{300}{\milli\volt}  \\ 
	\hline
\end{tabular} 
\addtocounter{table}{-1}
\renewcommand{\thetable}{A.\arabic{table} (suite)}
\caption{Table répertoriant les différents tests en faisceaux.}
\end{table}

\begin{table}
	\centering
	\begin{tabular}{|O|O|O|O|N}
		\hline 
		\rowcolor{gray!50}Test& \% SF6&\thead{GIF++\\octobre 2017} &Timing\\ 
		\hline 
		\thead{Section de la thèse}& \ref{SF66} (p.\pageref{SF66})& \ref{GIFF4} (p.\pageref{GIFF4})& \ref{TIMINGG} (p.\pageref{TIMINGG}) \\ 
		\hline 
		Date & mars--avril 2017 & octobre 2017 &   mai 2017   \\ 
		\hline 
		Lieu & \thead{Bât.904 CERN \\Lyon} & \thead{GIF++ \\H4 CERN}  & \thead{SPS \\H4 CERN}   \\ 
		\hline 
		\thead{Type de faisceaux}& Cosmique& $\mu$ (\textit{spill}) & $\mu$ (\textit{spill})  \\ 
		\hline 
		\thead{Intensité  de faisceaux}&$\sim$\SI{100}{\per\square\meter\per\second\per\steradian}&  $\sim$\num{1000}/spill& $\sim$\num{1000}/spill  \\ 
		\hline 
		\thead{Type de détecteur}&\thead{Basse résistivité\\ (Fixation mécanique)\\Basse résistivité \\\textit{Bakélite}}&\thead{Basse résistivité\\ Bakélite} &   Bakélite   \\ 
		\hline 
		\thead{Taille des détecteurs}& \thead{RE1/1 \\\num{32}$\times$\num{30}\si{\square\centi\meter}\\\num{32}$\times$\num{30}\si{\square\centi\meter}} & \thead{\num{32}$\times$\num{30}\si{\square\centi\meter}} &  \num{50}$\times$\num{50}\si{\square\centi\meter}  \\ 
		\hline 
		\thead{Nombre de détecteurs}&1/2& 2 &  2  \\ 
		\hline 
		\thead{Épaisseur des électrodes}&\thead{\SI{1}{\milli\meter} \\ \SI{1}{\milli\meter} \\ \SI{2}{\milli\meter}} & \thead{\SI{1}{\milli\meter} \\ \SI{2}{\milli\meter}}  &  \thead{\SI{1.4}{\milli\meter} \\ \SI{1.6}{\milli\meter}} \\ 
		\hline 
		\thead{Épaisseur du \textit{gap} }&\thead{\SI{1.2}{\milli\meter} \\ \SI{1.2}{\milli\meter} \\ \SI{2}{\milli\meter}} & \thead{\SI{1.2}{\milli\meter} \\ \SI{2}{\milli\meter}}  & \thead{\SI{1.4}{\milli\meter} \\ \SI{1.6}{\milli\meter}} \\ 
		\hline 
		Gaz & variable & CMS & CMS\\ 
		\hline 
		\thead{Type de chambre}& \textit{double gaps} & \textit{double gaps}  &  \textit{double gaps}  \\ 
		\hline 
		\thead{Type d'électronique}& HARDROC & HARDROC &  PETIROC2   \\ 
		\hline 
		\thead{Type de cellules \\de lecture}&\thead{\textit{strip} \\ \textit{strip} (cf.Fig~\ref{DoubleGap}) \\ \textit{strip} (cf.Fig~\ref{DoubleGap})} & \thead{\textit{strip} (cf.Fig~\ref{DoubleGap})}& \textit{strip}  \\ 
		\hline
		\thead{Tension de fonctionnement}& --- & \thead{\SI{6.8}{\kilo\volt} \\\SI{9.3}{\kilo\volt}} &    \\ 
		\hline
		\thead{Seuil de fonctionnement}& \thead{\SI{0.171}{\pico\coulomb}\\\SI{73}{\femto\coulomb}\\\SI{73}{\femto\coulomb}} & --- &  \\ 
		\hline
	\end{tabular} 
	\addtocounter{table}{-1}
	\renewcommand{\thetable}{A.\arabic{table} (suite)}
	\caption{Table répertoriant les différents tests en faisceaux.}
\end{table}
\captionsetup{list=yes}